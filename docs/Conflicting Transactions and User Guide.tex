\documentclass[12pt]{article}
\usepackage[utf8]{inputenc}
\usepackage[margin=0.8in]{geometry}
\usepackage{paralist}
\usepackage{blindtext}
\usepackage{hyperref}
\usepackage{chngcntr}
\usepackage{amsfonts,latexsym,amsthm,amssymb,amsmath,amscd,euscript}
\usepackage{enumitem}
\usepackage{caption}
\usepackage{algorithm}
\usepackage{algpseudocode}
\usepackage{amsmath}
\usepackage[table,xcdraw]{xcolor}
\usepackage{tikz}
\usepackage{parskip}
\usepackage{fancyhdr}


    \hypersetup{colorlinks=true,citecolor=blue,urlcolor =black,linkbordercolor={1 0 0}}

\usetikzlibrary{calc}

\renewcommand\thesubsubsection{\arabic{subsubsection}}

\newenvironment{statement}{\color[rgb]{1.00,0.00,0.50} {}}{}


\newcolumntype{M}[1]{>{\centering\arraybackslash}m{#1}}
\newcolumntype{N}{@{}m{0pt}@{}}
\algnewcommand{\algorithmicand}{\textbf{ and }}


\algnewcommand\algorithmicforeach{\textbf{for each}}
\algdef{S}[FOR]{ForEach}[1]{\algorithmicforeach\ #1\ \algorithmicdo}


\newlist{enums}{enumerate}{3}

\setlist[enums, 1]{label={(\arabic*)}, noitemsep}
\setlist[enums, 2]{label={-}, noitemsep}

\hypersetup{
    colorlinks=true,
    linkcolor=blue,
    filecolor=magenta,      
    urlcolor=black
    }

\title{Conflicting Transactions and User Guide}
% \author{Anirudh S. Kumar (2021517), Aakarsh Jain (2021507)}
\author{
    \\\vspace{0em} Anirudh S. Kumar \\\vspace{-0.5em}
    \footnotesize{Roll Number - 2021517}\\\vspace{-0.5em}
    \footnotesize{IIIT - Delhi}\\\vspace{-0.5em}
    \footnotesize{\href{mailto:anirudh21517@iiitd.ac.in}{\texttt{anirudh21517@iiitd.ac.in}}}
  \and
    \\\vspace{0em} Aakarsh Jain\\\vspace{-0.5em}
    \footnotesize{Roll Number - 2021507}\\\vspace{-0.5em}
    \footnotesize{IIIT - Delhi}\\\vspace{-0.5em}
    \footnotesize{\href{mailto:aakarsh21507@iiitd.ac.in}{\texttt{aakarsh21507@iiitd.ac.in}}} 
    \vspace{1em}
}

\renewcommand{\footrulewidth}{0.4pt}% Default \footrulewidth is 0pt
\renewcommand{\headrulewidth}{0pt}% Default \headrulewidth is 0.4pt


\date{\today}

\begin{document}
\maketitle

\pagestyle{fancy}
\fancyhf{}
\fancyfoot[L]{Anirudh S. Kumar}
\fancyfoot[C]{\thepage}
\fancyfoot[R]{Aakarsh Jain}



\section{Selling a Product from a Retailer}

Consider the transaction $T$ with two steps $S_1$ and $S_2$ as follows
\begin{itemize}[noitemsep]
    \item $S_1$: Reading the remaining product quantity from the retailer
    \item $S_2$: Changing the value of the remaining product quantity after order delivery.
\end{itemize}

\subsection{Conflict Serializable Schedule}

Now consider two such transactions, $T_1$ and $T_2$, in the following schedule.

\begin{enumerate}
    \item $S_1^{T_1}$ - $T_1$ reads from the database
    \item $S_2^{T_1}$ - $T_1$ writes to the database
    \item $S_1^{T_2}$ - $T_2$ reads from the database
    \item $S_2^{T_2}$ - $T_2$ writes to the database
\end{enumerate}

We can see that in this schedule, changing the ordering between the transactions does not affect the final value in the database. Therefore, this is an example of a conflict serializable schedule.


\subsection{Non-Conflict-Serializable Schedule}

Now consider the same two transactions, $T_1$ and $T_2$, in the following schedule.


\begin{enumerate}
    \item $S_1^{T_1}$ - $T_1$ reads from the database
    \item $S_1^{T_2}$ - $T_2$ reads from the database
    \item $S_2^{T_2}$ - $T_2$ writes to the database
    \item $S_2^{T_1}$ - $T_1$ writes to the database
\end{enumerate}

This is an example of a non-conflict serializable schedule. Consider the scenario where the initial quantity in the database was 20

\begin{enumerate}
    \item $S_1^{T_1}$ - $T_1$ reads from the database $\rightarrow$ \textbf{$T_1$ reads 20 }
    \item $S_1^{T_2}$ - $T_2$ reads from the database $\rightarrow$ \textbf{$T_2$ reads 20}
    \item $S_2^{T_2}$ - $T_2$ writes to the database $\rightarrow$  \textbf{$T_2$ buys 5 and updates the value to 15 in the database}
    \item $S_2^{T_1}$ - $T_1$ writes to the database $\rightarrow$  \textbf{$T_1$ buys 2 and updates the value to 18 in the database}
\end{enumerate}

If we were to reverse the order of operations, we would see that the final value in the database would be 15 instead. Therefore, this is a non-conflict serializable schedule.


\section{Buying a deleted product}

Consider two such transactions $T_1$ and $T_2$ each of them with two steps as follows :-
\begin{itemize}[noitemsep]
    \item $T_1$
    \begin{itemize}[noitemsep]
        \item $S_1$ - Reading the remaining product quantity of product $P$ from the retailer
        \item $S_2$ - Changing the value of the remaining product quantity $P$ after order delivery.
    \end{itemize}
    \item $T_2$
    \begin{itemize}[noitemsep]
        \item $S_1$ - Fetching the details of the product $P$
        \item $S_2$ - Removing it from the inventory list. 
    \end{itemize}
\end{itemize}

\subsection{Conflict Serializable Schedule}

Now consider two such transactions, $T_1$ and $T_2$, in the following schedule.

\begin{enumerate}
    \item $S_1^{T_1}$ - $T_1$ checks for availability of product $P$ in the databse
    \item $S_2^{T_1}$ - $T_1$ order the product $P$ if available 
    \item $S_1^{T_2}$ - $T_2$ fetches the product $P$ from the database
    \item $S_2^{T_2}$ - $T_2$ deletes the product $P$ from the database. 
\end{enumerate}

If we reverse the ordering of execution of $T_1$ and $T_2$, then $T_1$ will abort before it orders the product, so data integrity remains. 

\subsection{Non-Conflict-Serializable Schedule}


Now consider the same two transactions, $T_1$ and $T_2$, in the following schedule.

\begin{enumerate}
    \item $S_1^{T_1}$ - $T_1$ checks for availability of product $P$ in the databse
    \item $S_1^{T_2}$ - $T_2$ fetches the product $P$ from the database
    \item $S_2^{T_2}$ - $T_2$ deletes the product $P$ from the database. 
    \item $S_2^{T_1}$ - $T_1$ order the product $P$ if available 
\end{enumerate}

This is a case of a non-conflict-serializable schedule. In the first step, $T_1$ thinks that the product exists in the database and is available for ordering, but then $T_2$ fetches and deletes the product, causing the second step of $T_1$ to cause inconsistencies.

\section{User Guide}

\begin{itemize}
    \item \underline{Login/Signup} - Click Signup and enter the relevant details. After signup, click the Login button to log in to the site.
    \item \underline{Booking Product} - First Login/Signup. After that Go to Products $\rightarrow$ Pick a Category $\rightarrow$ Pick a Product $\rightarrow$ Add to cart $\rightarrow$ Cart $\rightarrow$ Checkout $\rightarrow$ Add Details $\rightarrow$ Place Order.
    \item \underline{Booking Lab Tests} - First Login/Signup. Afterwards, Go to Labs $\rightarrow$ Pick a test $\rightarrow$ Buy Now $\rightarrow$ Add Details $\rightarrow$ Place Order.
    
\end{itemize}

\end{document}