\documentclass{article}
\usepackage[utf8]{inputenc}
\usepackage[margin=0.9in]{geometry}
\usepackage{paralist}
\usepackage{blindtext}
\usepackage{hyperref}

\hypersetup{
    colorlinks=true,
    linkcolor=blue,
    filecolor=magenta,      
    urlcolor=black
    }

\title{Integrity Constraints and Data Generation - DrugVeda}
% \author{Anirudh S. Kumar (2021517), Aakarsh Jain (2021507)}
\author{
    \\\vspace{0em} Anirudh S. Kumar \\\vspace{-0.5em}
    \footnotesize{Roll Number - 2021517}\\\vspace{-0.5em}
    \footnotesize{IIIT - Delhi}\\\vspace{-0.5em}
    \footnotesize{\href{mailto:anirudh21517@iiitd.ac.in}{\texttt{anirudh21517@iiitd.ac.in}}}
  \and
    \\\vspace{0em} Aakarsh Jain \\\vspace{-0.5em}
    \footnotesize{Roll Number - 2021507}\\\vspace{-0.5em}
    \footnotesize{IIIT - Delhi}\\\vspace{-0.5em}
    \footnotesize{\href{mailto:aakarsh21507@iiitd.ac.in}{\texttt{aakarsh21507@iiitd.ac.in}}} 
    \vspace{1em}
}



\date{February 10, 2023}

\begin{document}

\maketitle

\section{Data Generation and Population}

\subsection{Data Generation}
\begin{itemize}
    \item Primary Data was generated through 2 sources
    \begin{itemize}
        \item For \textcolor{blue}{\texttt{products}}, \textcolor{blue}{\texttt{tests}} and \textcolor{blue}{\texttt{brands}}, we scrapped  \href{https://pharmeasy.in/}{\underline{Pharmeasy}}
        \item For \textcolor{blue}{\texttt{users}}, \textcolor{blue}{\texttt{retailers}}, \textcolor{blue}{\texttt{medical\_labs}} and \textcolor{blue}{\texttt{suppliers}}, we used 
        \href{https://mockaroo.com/}{\underline{Mockaroo}} to generate the data
    \end{itemize}

    \item Secondary data, like the orders and inventory for retailers, was made with the help of python.

    \item We made sure that whatever data we were generating would be consistent with the logic of our application while also having variety.
\end{itemize}

\subsection{Data Population}

All the data generated in the above process was stored inside JSON files, which were then parsed via Python to insert into the database.

\begin{itemize}
    \item We used SQLAlchemy as ORM to make the creation of the database process easier.
    \item We then used \texttt{mysql.connector} module in python to parse the JSON files and insert the data into the database
\end{itemize}

\section{Integrity Constraints}

For most of the tables, we tried to make a special ID column to ensure there is guaranteed one unique column among all the columns, which also served as the foreign key to different tables.
For example :-
\begin{itemize}
    \item Primary Key for \textcolor{blue}{\texttt{customers}} table is set to \texttt{CustomerID}
    \item Primary key for \textcolor{blue}{\texttt{appointments}} table is set to \texttt{AppointmentID}
\end{itemize}

However, certain tables do not have an explicit primary key but instead use a combination of foreign keys to identify a row uniquely. An example of this is the \textcolor{blue}{\texttt{inventory}} table, where every row has a uniq \texttt{BatchID} and \texttt{RetailerID} pair. 

\end{document}